%
% GKC, XIN YUAN, 2020
%

\cnappendixchapter{GKC库}

GKC库是按照分层的思想设计的。而各个部件也是按层次逐步添加的。

\section{分层}

GKC的程序编码分成四层,如表 \ref{tab:a-gkc:layers} 所示:

\begin{table}[h]
  \centering
  \caption{层次列表}\label{tab:a-gkc:layers}
\begin{tabular}{|l|p{3cm}|p{3cm}|p{3cm}|}
  \hline
  \textbf{层次} &  \textbf{描述} & \textbf{文件名规范} &  \textbf{程序命名规范}  \\
  \hline\hline
  用户  &  用户使用的接口  & 大写字母开头,大小写字母相间  &  大写字母开头,大小写字母相间  \\
  \hline
  用户内部层 & 用户内部使用的接口  & 下划线跟着大写字母开头,大小写字母相间  &  下划线跟着大写字母开头,大小写字母相间  \\
  \hline
  内部层  & 内部使用的接口  & 小写字母开头,全部小写,用下划线连接词  & 小写字母开头,全部小写,用下划线连接词  \\
  \hline
  系统内部层  & 操作系统相关的内部使用的接口  & 下划线跟着小写字母开头,全部小写,用下划线连接词  & 下划线跟着小写字母开头,全部小写,用下划线连接词  \\
  \hline
\end{tabular}

\end{table}

每个层都可以调用本层和其下层提供的类和函数接口。

\section{部件}

GKC的各个部件也是分层生成的。

\subsection{base}

在文件夹 public/include/base 下是最基础的程序,封装了操作系统相关的一些数据类型,
和一些方便的函数。文件 GkcDef.h 则包含了这些基本的头文件,并提供一些接口。
其他文件提供了常用的数据结构和各种不同类型工程需要的程序。

\subsection{sys}

文件夹 RT 下是运行时库的工程,产生共享的装配件,在 windows 下是动态链接库,
在 Linux 下是共享库对象。这些共享的装配件属于用户内部层,所以它们的对外公开接口文件是用下划线开头的命名,
提供的类和函数也是用下划线开头的命名。它们在实现共享装配件的程序中被使用。

文件夹 RT 下的 GkcSys 工程是基本的运行时库,提供了方便的函数,以及一些组件对象的创建函数。
它使用了base下的头文件来实现。RT/GkcSys/public 下的 \_GkcSys.h 提供对外的类和函数。

文件夹 public/include/sys 属于用户层,因而提供了对应的文件 GkcSys.h,
封装了用户内部层的 \_GkcSys.h 里的类和函数。该文件夹下的其他文件都是基于 GkcSys.h 和 base 下的头文件来实现的类和函数。

\subsection{compiler}

文件夹 RT 下的 GkcCompiler 工程是实现编译器的运行时库,包括词法分析、语法分析、抽象语法树等。
它使用了base,sys下的头文件来实现。RT/GkcCompiler/public 下的 \_GkcCompiler.h 提供对外的类和函数。

文件夹 public/include/compiler 属于用户层,因而提供了对应的文件 GkcCompiler.h,
封装了用户内部层的 \_GkcCompiler.h 里的类和函数。

\subsection{parser}

文件夹 RT 下的 GkcParser 工程是实现几种语言编译器的运行时库,包括wlang,won,wmark等。
它使用了base,sys,compiler下的头文件来实现。RT/GkcParser/public 下的 \_GkcParser.h 提供对外的类和函数。

文件夹 public/include/parser 属于用户层,因而提供了对应的文件 GkcParser.h,
封装了用户内部层的 \_GkcParser.h 里的类和函数。

\subsection{ui}

文件夹 public/include/ui 下提供编写界面程序需要的类和函数,可以使用base,sys下的头文件。
