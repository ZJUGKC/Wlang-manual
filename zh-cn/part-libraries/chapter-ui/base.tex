%
% Basic topics, XIN YUAN, 2020
%

\chapter{ATL~和~WTL~的基本原理}

Windows~操作系统自诞生以来,就出现了为数不少的应用
程序框架,以方便其编写。最早是~Windows API~编程,
使用~C~语言,其后~Borland~推出了~OWL~框架,微软也
提出了~MFC~框架\index{MFC}。很快,~MFC~取代了~OWL~,
同时用~VB~进行快速开发。Borland~在修改的~Pascal~
语言的基础上,提出了~Delphi~语言,使用~VCL~,
风靡一时。随着微软在操作系统中大量使用~COM~技术
构造系统级程序,Delphi~发展变慢,而网络的迅猛发展,
使得~ActiveX~之类的网络小程序变得普遍起来,这就
需要新的编程框架来编写出功能强大且尺寸小的组件。
ATL~正是为这个目的设计的。它和~MFC~生成的程序的
臃肿形成了鲜明的对比。而~WTL~是~ATL~的扩展,是
为了更方便地编写~windows~桌面程序而产生的。尽管
未得到微软的官方支持,但仍作为一个开源库在微软
网站上和~sourceforge~社区得到支持和更新。同时期
也涌现出其他各种框架程序,如微软的~.NET~框架,
跨平台的~Qt~,~WxWidget~等,后面这些框架基本延续
~MFC~和~VCL~的编程思路,不象~ATL~利用大量的~C++~
高级特性,因此它们生成的代码仍然臃肿。而~ATL~和
~WTL~以简单的结构,优雅的语法,生成短小的可执行
文件,吸引了相当数量的编程爱好者,为其添砖加瓦。

\section{Windows~编程介绍}

Windows~系统的核心是图形交互~(GUI)~。一般的~windows~
编程主要包含以下内容:
\begin{itemize}
    \item 窗口
    \item 对话框和属性页
    \item 通用控制
    \item 子类化和超类化
    \item 加速键
    \item 位图
    \item 其他资源
    \item 框架窗口的基本元素
    \item 鼠标和键盘输入
    \item 系统级编程如进程线程串口等
    \item COM~和~ActiveX
    \item 3D~图形
\end{itemize}

具体细节可参考任何一本讲~windows~基本编程的书
\cite{wincxsj:2004}。

\section{ATL~的封装}

ATL~是为了编写轻量级的~COM~组件,包括~ActiveX~控制
而开发出的基于模板的应用程序框架。ATL~用到的~C++~
高级特性包括模板、多重继承、虚函数表初始化技巧、
thunk~技术、type cast~等。其中模板和多重继承是有争议
的~C++~技术,被认为过于复杂,是编程大忌,在其他的
~C++~的~windows~框架库中就没有按这个思想去编写。但
正是这些技术,使得~ATL~生成的可执行文件大大减少,
几乎没有冗余代码,并且和直接调用~Windows API~一样
高效。

ATL~为我们提供了大量~windows~底层的薄封装。几乎所有
需要封装的内容,都已经封装好。不仅如此,它们还通过
成员函数的形式提供与~API~同名的或者具有更友好名字的
函数,然后直接调用~API~,在析构方法中,释放已分配的
系统资源。以下各小节介绍~ATL~典型的封装类和工具类。

\subsection{组件接口}

接口智能指针~CComPtr$<$T$>$~提供了对~COM~接口操作的
支持。模板参数~T~是接口。主要方法有创建组件的一个接口
~CoCreateInstance~,直接访问接口方法的操作符~-$>$~,
释放方法~Release~,析构方法能自动释放接口。

类~CComVariant~封装了变体结构~VARIANT~和相关的操作函数。
使用该类,可以方便地转换变体数据。

类~CComBSTR~封装了~BSTR~类型的数据,即双字节字符串,在
~windows~中即~Unicode~字符串。因为接口方法均使用该类型
描述字符串。该类可以方便地转换~Unicode~字符串和
~ANSI~字符串。

\subsection{系统级别的类}

类~CHandle~封装了~windows~最基本的底层操作对象,它提供
了对句柄的基本处理方法。

类~CCritSecLock~封装了线程同步对象,关键段
~CRITICAL\_SECTION~,因为是在同一个进程内提供加锁解锁
功能,因此效率较高。

\subsection{全局变量}

ATL~工程一般只有一个全局变量~\_Module~,是~CComModule~
类型的。该变量管理整个进程模块的基本操作。完成工程的
过程中,尽量不要增加自己的全局变量,特别是需要依赖
~\_Module~初始化之后才能构造的全局变量。可以把自己的
全局变量做成~CComModule~类的派生类的成员变量,\_Module~
就是该派生类的变量。也可以把自己的全局变量做成一些类的
静态成员。注意它们的初始化不能有相互依赖性。除非有十足
的把握去自己定义全局变量,否则应尽量避免使用全局变量。

\subsection{Windowing~封装}

ATL~设计了一个窗口句柄~HWND~的封装类,CWindow~,类似于
~Handle~的封装类,仅有一个数据成员,即窗口句柄~m\_hWnd~,
没有虚函数。所以这个类和窗口句柄占用同样大小的空间,
某种意义上是等价的,因此可以直接作为函数参数传递而不会
损失效率,因为它只要把一个~HWND~数据复制到栈上即可。
CWindow~的析构方法并不销毁窗口,也就是说,CWindow~对象
离开了作用域之后,并不会造成窗口的销毁。

不难看出,CWindow~类没有处理~windows~消息。对于应用程序
来说,需要自己处理窗口的某些消息。ATL~提供了~CWindowImpl~
类为用户提供窗口消息的定制处理。

Windows~下~SDK~编程的窗口消息处理,是采用窗口回调函数的
方法进行的。一般地,每个不同类型的窗口都有一个回调函数
来处理窗口的消息。使用面向对象的程序设计,更自然的方式
是每个不同类型的窗口对应一个窗口类,使用类的方法来处理
窗口消息。这样就引出了一个问题,即如何将窗口句柄和类的
实例联系起来,从而能在窗口的回调函数中调用类的方法来
完成消息映射处理。在~MFC~框架中使用的是一个全局的
哈希映射表,因此,在每次调用窗口回调函数时,都要根据
传递进来的窗口句柄查找得到对应的窗口类的实例指针,再
调用它的虚方法来完成消息处理。这样做显然十分耗时。而
~ATL~使用的则是完全不同的一种方法,称为~thunk~方法。
下面给出这个方法的简要描述,细节请自己读~ATL~的源代码
以及文献~\cite{ATLjsnm:2003}~。

ATL~的窗口句柄和类实例映射的实现思想是在调用窗口的
回调函数时,把函数调用的栈上的第一个参数,即窗口句柄
~HWND~用该窗口回调函数对应的窗口类的实例指针~pThis~
来替换,然后再跳转到真正的窗口回调函数的代码去执行。
这段替换代码就称为~thunk~。它是一小段机器码,可以用
结构表示如下:
\ttfamily
\begin{lstlisting}
#pragma pack(push,1)
struct _stdcallthunk
{
    DWORD   m_mov;
    // mov dword ptr [esp+0x4], pThis (esp+0x4 is hWnd)
    DWORD   m_this;         //
    BYTE    m_jmp;          // jmp WndProc
    DWORD   m_relproc;      // relative jmp
    BOOL Init(DWORD_PTR proc, void* pThis)
    {
        m_mov = 0x042444C7;  //C7 44 24 0C
        m_this = PtrToUlong(pThis);
        m_jmp = 0xe9;
        m_relproc = DWORD((INT_PTR)proc
            - ((INT_PTR)this+sizeof(_stdcallthunk)));
        // write block from data cache and
        //  flush from instruction cache
        FlushInstructionCache(GetCurrentProcess(),
                        this, sizeof(_stdcallthunk));
        return TRUE;
    }
    //some thunks will dynamically allocate the
    //memory for the code
    void* GetCodeAddress()
    {
        return this;
    }
    void* operator new(size_t)
    {
        return __AllocStdCallThunk();
    }
    void operator delete(void* pThunk)
    {
        __FreeStdCallThunk(pThunk);
    }
};
#pragma pack(pop)
\end{lstlisting}
\CJKfamily{song}

这时在窗口类的消息回调函数(设计成静态方法)中,
第一个参数就是窗口类实例的指针,利用它调用一个
虚函数,就可以使用类的方法来处理窗口消息了。

类~CWindowImpl$<$T, TBase, TWinTraits$>$~是一个
模板类,模板参数~T~是最终的派生类,TBase~是基类,
这里要求是~CWindow~或者是~CWindow~的派生类,如
那些标准控制类~CButton~等,当然也可以是
~CWindowImpl~定制的派生类,但这样设计概念就比较
复杂,可读性差。TWinTraits~是窗口风格类,可以用
~CWinTraits$<$t\_dwStyle, t\_dwExStyle$>$~
定制得到。这个模板类的功能就是提供了~thunk~方法,
建立窗口句柄和类方法之间的映射关系。

CWindowImpl~定义了~thunk~数据结构的成员变量,
因此,对于每个~CWindowImpl~定制的派生类的实例,
都有一个自己的~thunk~,即这段机器码。CWindowImpl~
的~Create~方法是用来创建窗口的,它首先注册一个
窗口类,把窗口回调过程设置为静态方法
~StartWindowProc~。注意每个最后的定制类都有
各自的静态方法,相同的定制类的不同实例才共享
一个静态方法。然后初始化~thunk~成员数据结构,
同时填充一个描述窗口信息的成员结构,其中包含有
类的~this~指针值、当前线程~ID~,并把这个成员结构
的地址加入到全局变量~\_Module~的一个成员链表中。
随后调用~CreateWindowEx~这个~API~函数。在此~API~
函数的执行过程中,将发出~WM\_CREATE~窗口事件,
直接调用窗口回调函数,执行完成后才返回~API~函数。
此时正是调用静态方法~StartWindowProc~。在该方法中,
首先从~\_Module~的成员链表中取得刚才加入的
和当前线程~ID~相同的要创建的窗口信息结构地址,该
地址即是该窗口类的窗口信息成员结构地址,而该类的
内存实例在整个创建窗口的过程中都是存在的,因此是
有效地址。又因为同一个线程一次只能创建一个窗口,
所以这个成员结构要么不存在,要么是唯一的。其次用
成员结构信息数据中的~this~值和类的另外一个静态方法
~WindowProc~的地址初始化~thunk~成员结构,并把窗口
回调过程设置为~thunk~成员结构内存实例的首地址,
最后调用~WindowProc~完成窗口消息的其他处理。
窗口创建完成后,以后若有该窗口类的窗口事件发生,
则首先执行~thunk~内存映像的代码,执行结果是用机器码
中的该窗口类实例的指针~pThis~替换栈上函数参数列表的
第一个参数,再跳转执行静态方法~WindowProc~。此时,
对于静态方法~WindowProc~来说,第一个参数已不是~HWND~,
而是窗口类实例指针~pThis~了。利用此指针可调用虚函数
~ProcessWindowMessage~,用户可重载此虚函数,就可用
窗口类的方法来处理窗口消息了。ATL~提供了一系列宏,
有~BEGIN\_MSG\_MAP~,END\_MSG\_MAP~,MESSAGE\_HANDLER~,
COMMAND\_HANDLER~,NOTIFY\_HANDLER~等来隐含编写虚函数
~ProcessWindowMessage~,把具体的窗口消息~ID~映射到类的
方法上去,用户最后只要实现这些类的方法即可。

上述窗口类窗口的创建和填充~thunk~需要分两步,涉及到
两个静态方法,是因为创建窗口前,窗口句柄还不存在,
也无法替换其窗口回调函数地址,而调用~API~创建窗口时
就要发出窗口事件,调用窗口回调函数,所以就只能用两步
方法,在第一次调用窗口回调函数~StartWindowProc~时
正式初始化~thunk~,并将以后的窗口回调函数调用转向到
第二个静态方法~WindowProc~。

窗口类的子类化方法~SubclassWindow~的实现也是类似的
过程。不过因为要子类化的窗口句柄已经存在,因此不需要
上述的两步方法,而是直接用~this~值和~WindowProc~的
地址来初始化~thunk~,从而转向到用户的类方法。

总的来说,这个技术的实质就是每个窗口类的实例都有
各自的一段~thunk~机器码,用各自的实例的指针替换类的
静态方法的调用参数,再利用一个虚函数来调用类的方法
实现对窗口消息的处理。显然,这个方法比~MFC~的哈希表
映射查找的效率要高,且最终生成的可执行代码要小得多。

对于对话框类型的窗口,ATL~提供了~CDialogImpl~模板类,
可以用~Create~方法创建无模式对话框,或者~DoModal~
方法建立有模式对话框。其中也采用了类似的~thunk~技术
来把对话框过程映射到对话框类的方法上。具体细节可
自己查看源代码。

这段机器码在何种虚存页面中执行,取决于窗口类的变量
是栈上变量还是堆上变量。在函数或者类方法内定义的
局部变量是栈上变量,使用~new~操作符产生的类变量是
堆上变量。Windows~平台下已经保证这两类变量所在虚存
页面都是可执行的。在移植到其他平台时,要注意~thunk~
所在页面必须是可执行的。在高版本的~ATL~的代码实现中,
通过自动动态分配具有可执行属性的页面,来容纳每个窗口类
实例的~thunk~代码数据。

\subsection{Windows~控制}

Windows~已经提供了一批标准控制,如按钮、列表、树视图等。
它们也是窗口,不过它们的行为已经由~Windows~实现好了。
ATL~没有对这些标准控制进行封装,不过它提供了对现有控制
进行功能扩展的方法,即子类化和超类化的方法。

子类化方法是替换现有窗口控制的窗口回调函数,由用户来
处理某些窗口消息,并根据需要最后调用原来的窗口控制的
窗口回调函数。因此可以使用~CWindowImpl~模板类来实现
子类化,父类模板参数~TBase~可用~CWindow~。使用窗口类别
的定义宏~DECLARE\_WND\_CLASS~或者~DECLARE\_WND\_CLASS\_EX~
来定义窗口类别。消息处理使用~BEGIN\_MSG\_MAP~系列的宏。
类设计完成后,调用类的~SubclassWindow~方法来替换窗口控制
的回调过程。

超类化方法也是替换现有窗口控制的窗口回调函数,由用户
来处理某些窗口消息,并根据需要最后调用原来的窗口控制
的窗口回调函数。不过,超类化需要明确提供一个窗口类别
的名字,要获取已有窗口控制的类别的信息,稍加修改再
创建新窗口类别。使用该窗口类时,不再调用
~SubclassWindow~方法,而是用~Create~方法直接创建
窗口控制。这样做的意义在于创建了新的扩展功能的窗口
类别,可供应用程序的其他部分直接利用此窗口类别来创建
窗口,便于协作编程。同样可使用~CWindowImpl~模板类来
实现超类化,父类模板参数~TBase~可用现有控制的包装类,
它们应是~CWindow~的派生类。使用窗口类别的定义宏
~DECLARE\_WND\_SUPERCLASS~来定义窗口类别,给出
窗口控制原来的类别名和新的类别名。消息处理使用
~BEGIN\_MSG\_MAP~系列的宏。类设计完成后,调用类的
~Create~方法来创建扩展功能的窗口控制。

显然,可以编写一个窗口类,既可作为子类化使用,也可作为
超类化使用。使用~CWindowImpl~模板类,用宏
~DECLARE\_WND\_SUPERCLASS~定义超类,同时重载
~SubclassWindow~和~UnsubclassWindow~方法,实现
窗口消息~WM\_CREATE~和~WM\_DESTROY~的处理。然后加上
用户自己的代码,就可实现窗口控制的扩展功能。

另一种子类化和超类化的方法是使用~CContainedWindow~类,
它可以将窗口控制的消息处理转向到父窗口,通过在父窗口
中处理这些消息达到扩展窗口控制的功能的目的。

ActiveX~控件是基于~COM~组件架构的窗口控制,可以被其他
语言开发的应用程序如~Visual Basic~或者网页~ASP~脚本所
方便地使用。ATL~原本就是为设计小尺寸的可执行映像的~COM~
组件而诞生的,所以为设计~ActiveX~控件已经提供了大量的
模板类的支持。有关编程技术可参考~ATL~的文献。

ActiveX~控件分为有内部窗口和无内部窗口两种。无窗口的
控件将使用存放该控件的容器的窗口来绘图。为了能在桌面
程序或者其他~ActiveX~控件中使用~ActiveX~控件,需要
一个容器类,能放置一个或多个~ActiveX~控件,而这个容器
类同时也是窗口类。这个技术称为~ActiveX Hosting~技术。
ATL~提供了~CAxWindow~和~CAxDialogImpl~类来帮助实现能
容纳~ActiveX~控制的窗口类。

\subsection{窗口的反射机制}

窗口可以想象成桌面上的纸片,它们可以相互重叠嵌套形成
层次结构。显然,窗口之间如何传递信息,是一个主要的
编程问题。一般地,Windows~提供了~PostMessage~和
~SendMessage~两个~API~函数以向其他窗口发送消息进行
通讯。而在窗口编程中,窗口间最重要的关系是父子关系。
一般地,窗口控制通过向父窗口发送~WM\_COMMAND~和
~WM\_NOTIFY~消息来通知父窗口,子窗口控制发生的事件,
由父窗口根据用户需要定制处理。一般调用~SendMessage~
来通知父窗口,因此是直接调用了父窗口的窗口函数,要等
用户处理完这条通知消息之后,子窗口控制继续执行其他的
处理。

子窗口控制的通知消息处理一般放在父窗口中。但如果消息
处理和子窗口本身紧密相关,这样编写程序就不是很方便。
ATL~引入了反射的概念来让子窗口控制类处理通知消息。其
原理是父窗口使用一个消息宏~REFLECT\_NOTIFICATIONS~,
处理~WM\_COMMAND~和~WM\_NOTIFY~消息,取出消息参数中
的窗口句柄,即为发出通知消息的子窗口控制的句柄。然后
将消息号整数值加上一个常数,再向子窗口控制发回该消息。
此时~WM\_COMMAND~和~WM\_NOTIFY~消息变成了~OCM\_COMMAND~
和~OCM\_NOTIFY~消息。程序员可以在子窗口控制类中处理
这两条消息。这就是消息反射的过程。

用户可以在子窗口控制类中使用~REFLECTED\_~开头的
消息映射宏来处理子窗口控制经父窗口反射回来的通知消息。
从而可以在子窗口控制类中方便地扩展控件功能。

\subsection{Mix-in class~的设计模式}

ATL~提出了一种新的程序设计模式,即混合类~Mix-in class~
模式。该模式使用了~C++~的两个高级特性:模板类和多重继承。
这两个特性也是~C++~中受到争议的特性。一般认为,使用不同
的模板参数实例化模板类时,每次都生成了新的类,使得程序
最终将生成庞大的类库,从而加重了编译器的负担。而多重继承
使得派生类将拥有不止一个的父类,对于父类中存在的同名方法,
就无法区分是调用哪一个方法。父类中若含有虚函数,则情况
更为复杂,除了要区分是哪一个父类的方法外,还跟继承路径
有关,这就是典型的菱形继承问题。正因为如此复杂,所以目前
相当多的通用~C++~类库,包括商业用途的类库,都选择了单根
的继承结构,即所有的类都只有一个父类,且都从同一个基本类
派生而来。这样做虽然在编程的概念上得到了简化,但是在生成
可执行映像时链接进了大量不必要的代码,所以可执行文件显得
十分庞大,在网络环境中需要传输程序到客户浏览器的场合下,
这样的编程方式就显得不够灵活。

ATL~通过使用模板类和多重继承,做到了真正的代码复用。同时
因为模板类在编译过程中,只有用到的方法才会被实例化,这样
最终的可执行文件不再包含无用的代码,因此程序的大小得到
大幅度的减少,达到令人吃惊的地步。典型的~COM~组件文件只需
~30K~左右的磁盘空间。这样大大利于网络上~ActiveX~小程序的
快速传输。

下面给出。。。

。。。
通过多重继承引入模板类的功能
。。



\subsection{编程风格}

inline 的问题 \_\_forceinline noinline

\ttfamily
\begin{lstlisting}
__declspec( noinline )

#ifdef _ATL_DISABLE_NO_VTABLE
#define ATL_NO_VTABLE
#else
#define ATL_NO_VTABLE __declspec(novtable)
#endif

#ifdef _ATL_DISABLE_FORCEINLINE
#define ATL_FORCEINLINE
#else
#define ATL_FORCEINLINE __forceinline
#endif

#ifdef _ATL_DISABLE_NOINLINE
#define ATL_NOINLINE
#else
#define ATL_NOINLINE __declspec( noinline )
#endif

ATL_NOINLINE inline
ATL_NOINLINE

#ifdef _ATL_DISABLE_DEPRECATED
#define ATL_DEPRECATED(_Message)
#else
#define ATL_DEPRECATED(_Message) __declspec( deprecated(_Message) )
#endif
\end{lstlisting}

多重继承规则 不使用虚拟继承
static\_cast<T*>(this)

 写单个头文件

类的交叉引用

初始化静态变量

ATL\_NO\_VTABLE

用到才实例化

其他工具类
内存Allocator IAtlMemMgr C...
智能指针 CAutoPtr CAutoVectorPtr CAutoStackPtr
CImage GDI+
CAtlFile CAtlTemporaryFile CAtlFileMaping
AtlPath Path 函数
数据结构 集合类 CAtlArray CAtlList CAtlMap(HashTable) 可能会抛出异常
CSimpleArray CSimpleMap
CRBTree


ATL Server 服务程序 http web service XML web service
socket for SOAP SMTP HTTP

thread: AtlCreateThread COM线程模型
CThreadPool CWorkerThread IO完成端口模型
atlsync 提供了同步对象

CString类
CRect等类

ANSI UNICODE BSTR OLESTR转换宏。
C开头的宏实际是类,且它的构造函数会抛出任何异常,
因此使用类变量,并捕获异常。或者在参数调用中使用构造函数,
产生临时对象,调用后自动释放的办法。另有一套使用
USES\_CONVERSION A2T等在栈上分配内存的字符串转换宏,在函数
退出时自动释放栈上空间。

\section{WTL~的封装}

在~ATL~封装的基础上,WTL~进一步对窗口、控制和~GDI~进行
了封装。除此以外,还封装了一些有用的工具类。封装的基本
特征如下:
\begin{itemize}
    \item 几乎全部~C++~特性
    \item 对~windows~通用控制进行了薄封装
    \item 一个小巧但强大的应用程序框架,并取消了
文档~/~视架构
    \item 打印和打印预览的支持
    \item 移植~MFC~的类~CPoint~,CRect~,CSize~,
CString~等。
    \item 提供了消息破解宏
\end{itemize}

其中有些封装如~CString~和~ATL~产生了重复,可以根据不同
的需要去使用,需在类名前加上名字空间说明,如~WTL::CString~
或者~ATL::CString~。

表~\ref{tab:basic:files}~给出了~WTL~提供的头文件的功能
说明。

\begin{table}[h]
  \centering
  \caption{文件说明}\label{tab:basic:files}
\begin{tabular}{|l|l|}
  \hline
  % after \\: \hline or \cline{col1-col2} \cline{col3-col4} ...
  \textbf{文件名} & \textbf{功能} \\
  \hline\hline
  atlapp.h & 消息循环接口及应用模块类 \\
  \hline
  atlcrack.h & 消息破解宏 \\
  \hline
  atlctrls.h & 标准控制和通用控制的类 \\
  \hline
  atlctrlw.h & Command Bar~类 \\
  \hline
  atlctrlx.h & 位图按钮、带检查框的~ListView~,和其他控制 \\
  \hline
  atlddx.h & 对话框和窗口中的控制和变量间的数据交换 \\
  \hline
  atldlgs.h & 通用对话框类,属性页类 \\
  \hline
  atlfind.h & 查找/替换通用对话框类 \\
  \hline
  atlframe.h & 框架窗口类,多文档,UI~更新类 \\
  \hline
  atlgdi.h & DC~类,GDI~类 \\
  \hline
  atlmisc.h & CPoint, CRect, CSize, CString~等类 \\
  \hline
  atlprint.h & 打印和打印预览 \\
  \hline
  atlres.h & 标准资源~ID \\
  \hline
  atlresce.h & Windows CE~下的标准资源~ID \\
  \hline
  atlscrl.h & 窗口滚动的支持 \\
  \hline
  atlsplit.h & 切分窗口 \\
  \hline
  atltheme.h & 窗口方案和动画 \\
  \hline
  atluser.h & 菜单类、像标类、资源类 \\
  \hline
  atlwince.h & Windows CE~下的窗口支持类 \\
  \hline
  atlwinx.h & 窗口增强类 \\
  \hline
\end{tabular}

\end{table}

GDI~类


  template <class TBase>

  class CButtonT : public TBase
  typedef CButtonT<ATL::CWindow>   CButton;

说到现在,如果我要使用呢?MFC有一系列的类,可以让我们
很容易地使用Windows的标准控件,而SDK下面,很多时候不得不使用直接对控件
的窗口句柄发送消息的做法,如果你用SDK写过带有ListView Control的程序,
估计你会对一大堆的LVM\_xxxx深恶痛绝,当然,你可以用WindowsX.h当中定义的
一些可爱的宏,但是看起来总是有点让人感到不是那么爽。幸运的是,WTL已经像
MFC那样,把所有的Windows标准控件进行了封装,并且提供了更加友好的接口函数
来对控件进行操作。还是随便从头文件当中摘抄一些出来吧:

WTL~提供了对~Windows~的薄封装,在编写时仍按~Win32 SDK~的
方式进行程序设计,使用的结构、常数仍然是~Win32~中的结构
和常数,因此其效率和直接使用~SDK~相近,而在编程的思路上
更接近人的抽象思维,相对较自然。
