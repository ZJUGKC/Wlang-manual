%
% Basic Data Types, XIN YUAN, 2020
%

数据类型分为值类型和引用类型两种。

\subsection{值类型}

变量是存储信息的基本单元;是计算机内存中的一个存储空间。

值类型分为几下几种:

\begin{itemize}
\item{简单类型}
\item{结构类型}
\item{枚举类型}
\item{联合体和位段类型}
\end{itemize}

简单类型即纯量类型,是直接由一系列元素构成的类型。包括整数、
布尔、字符、实数四种。

\subsubsection{整型}

计算机上的整数是有范围的。表~\ref{tab_int}~ 就是整数类型的表示。

\begin{table}[h]
  \centering
  \caption{整数类型}\label{tab_int}
  \vspace{1.5ex}
\begin{tabular}{|l|l|l|}
  \hline
  % after \\: \hline or \cline{col1-col2} \cline{col3-col4} ...
  \textbf{数据类型} & \textbf{特征} & \textbf{取值范围} \\
  \hline\hline
  char & 有符号~8~位整数 & -128~127 \\
  \hline
  uchar & 无符号~8~位整数 & 0~255 \\
  \hline
  short & 有符号~16~位整数 & -32768~32767 \\
  \hline
  ushort & 无符号~16~位整数 & 0~65535 \\
  \hline
  long & 有符号~32~位整数 & -2147483648~2147483647 \\
  \hline
  ulong & 无符号~32~位整数 & 0~4294967295 \\
  \hline
   &  & -9223372036854775808 \\
  \raisebox{1.3ex}[0pt]{int64} & \raisebox{1.3ex}[0pt]{有符号~64~位整数}
  & ~9223372036854775807 \\
  \hline
  uint64 & 无符号~64~位整数 & 0~18446744073709551615 \\
  \hline
  int128 & 有符号~128~位整数 & $-2^{63}$~$2^{63}-1$ \\
  \hline
  uint128 & 无符号~128~位整数 & $0$~$2^{64}-1$ \\
  \hline
\end{tabular}

\end{table}

此外定义两个关键字:~int~和~uint~,用来表示机器字长的整数。不同
的机器通用寄存器的字长不同,利用这两个类型的变量可以表示不同目标
机器上的通用寄存器整数,从而实现跨硬件平台。

整数的内存表示只需要注意不同硬件上的大端和小端问题。

以上整数的正则表达式为:

\newcommand{\clrsmall}[1]{\textcolor{magenta}{#1}}
\newcommand{\clrbig}[1]{\textcolor{blue}{#1}}

\begin{Verbatim}[frame=single, commandchars=*()]
    *clrsmall(digit)     [0-9]
    *clrsmall(hex_digit) [0-9a-fA-F]
    *clrbig(TK_PINUM)   {digit}+
    *clrbig(TK_PHEXNUM) 0[xX]{hex_digit}+
    *clrbig(TK_KEY_CHAR)     "char"
    *clrbig(TK_KEY_UCHAR)    "uchar"
    *clrbig(TK_KEY_SHORT)    "short"
    *clrbig(TK_KEY_USHORT)   "ushort"
    *clrbig(TK_KEY_LONG)     "long"
    *clrbig(TK_KEY_ULONG)    "ulong"
    *clrbig(TK_KEY_INT64)    "int64"
    *clrbig(TK_KEY_UINT64)   "uint64"
    *clrbig(TK_KEY_INT128)   "int128"
    *clrbig(TK_KEY_UINT128)  "uint128"
    *clrbig(TK_KEY_INT)      "int"
    *clrbig(TK_KEY_UINT)     "uint"
\end{Verbatim}

\subsubsection{布尔型}

布尔型是用来表示“真”和“假”的概念的。采用关键字~true~和~false~来
表示。和~C~与~C++~不同,~true~不能用其他非零值代替。整数类型不能
转换为布尔型。用关键字~bool~表示该数据类型。实现的方式是针对同一
个作用域,将所有~bool~变量集中到一个或多个~int~变量,每个~bool~变
量占用一个位。

正则表达式:

\begin{Verbatim}[frame=single, commandchars=\\\{\}]
    \textcolor{blue}{TK_KEY_BOOL}   "bool"
    \textcolor{blue}{TK_KEY_TRUE}   "true"
    \textcolor{blue}{TK_KEY_FALSE}  "false"
\end{Verbatim}

\subsubsection{字符型}

字符分为~char~和~wchar~两种,分别表示~ANSI~字符和~UNICODE~字符。
其中~wchar~的位数根据系统的不同而不同。~Windows~下是~16~位的,某些
~X~系统下是~32~位的。它们需要显式转换为其他类型。

如下代码对字符赋值:

\lstsetwlang

\ttfamily
\begin{lstlisting}
    char c = 'A';
    wchar c = L'A';
    char c = '\x05';
    wchar c = L'\x05';
    wchar c = '\x0032';
\end{lstlisting}
\CJKfamily{song}

其中前缀~L~表示其后字符为~UNICODE~字符;~$\backslash$x~后跟一个~16~
进制数表示字符。

转义符~``$\backslash$''~用来表示特殊的控制字符,如表~\ref{tab_esc}~
所示:

\begin{table}[h]
  \centering
  \caption{转义符}\label{tab_esc}
  \vspace{1.5ex}
\begin{tabular}{|l|l|}
  \hline
  % after \\: \hline or \cline{col1-col2} \cline{col3-col4} ...
  \textbf{转义符} & \textbf{含义} \\
  \hline\hline
  $\backslash$' & 单引号 \\
  \hline
  $\backslash$'' & 双引号 \\
  \hline
  $\backslash\backslash$ & 反斜杠 \\
  \hline
  $\backslash$0 & 空字符 \\
  \hline
  $\backslash$a & 感叹号 \\
  \hline
  $\backslash$b & 退格 \\
  \hline
  $\backslash$f & 换页 \\
  \hline
  $\backslash$n & 新行 \\
  \hline
  $\backslash$r & 回车 \\
  \hline
  $\backslash$t & 水平~tab~ \\
  \hline
  $\backslash$v & 垂直~tab~ \\
  \hline
  $\backslash$x & 后面的数字是~16~进制数 \\
  \hline
\end{tabular}

\end{table}

正则表达式:

\begin{Verbatim}[frame=single, commandchars=*\#\~]
    *clrsmall#ansi_char~      ([^'\"\n]|\\(['\"\\0abfnrtv]|
                             x{hex_digit}{hex_digit}))
    *clrsmall#wide_char~      (\\x{hex_digit}{hex_digit}
                              {hex_digit}+
                             |[x80-xff][x00-xff])
    *clrbig#TK_KEY_WCHAR~   "wchar"
    *clrbig#TK_CONST_CHAR~  '{ansi_char}'
    *clrbig#TK_CONST_WCHAR~ ((L'{ansi_char})|('{wide_char}))'
\end{Verbatim}

\subsubsection{实型}

实数类型包括浮点和十进制数两种类型。

\textbf{浮点类型}

浮点包括单精度~(float)~,双精度~(double)~和长精度~(longdbl)~三种
类型。计算机对浮点数的运算速度大大低于对整数的运算速度。

\begin{itemize}
\item{单精度}

占~32~位,取值范围:~$\pm 1.5\times 10^{-45}$~到
~$\pm 3.4\times 10^{38}$~,小数精度~7~位数。

内存格式:待补充。。。。。。
无效数的表示:

\item{双精度}

占~64~位,取值范围:~$\pm 5.0\times 10^{-324}$~到
~$\pm 1.7\times 10^{308}$~,小数精度~15~到~16~位数。

内存格式:待补充。。。。。。
无效数的表示:

\item{长精度}

占用~80~位,

内存格式:待补充。。。。。。
无效数的表示:

\end{itemize}

\textbf{十进制类型}

十进制数~(decimal)~主要用来表示金融和货币的数字计算。它是高精度、
~128~位的数字类型,表示范围为~$\pm 1.0\times 10^{-28}$~到
~$\pm 7.9\times 10^{-28}$~,有效数字~28~至~29~位。运算结果准确到
~28~个小数位。

十进制类型的计算机内存格式:待补充。。。。。。
无效数的表示:

实数常量代码表示:

\ttfamily
\begin{lstlisting}
    float a = 1.0f;
    double a = 1.0;
    longdbl a = 1.0ld;
    decimal a = 1.0m;
\end{lstlisting}
\CJKfamily{song}

其中后缀~f~表示单精度浮点数,~ld~表示双精度浮点,~m~表示十进制数。

正则表达式:

\begin{Verbatim}[frame=single, commandchars=^\#\~]
    ^clrsmall#num_exp~      [eE][+-]?{digit}+
    ^clrsmall#float_num~    ({digit}+"."{digit}*({num_exp})?|
                       {digit}+{num_exp}|
                       "."{digit}+({num_exp})?)
    ^clrbig#TK_KEY_FLOAT~       "float"
    ^clrbig#TK_KEY_DOUBLE~      "double"
    ^clrbig#TK_KEY_LONGDBL~     "longdbl"
    ^clrbig#TK_KEY_DECIMAL~     "decimal"
    ^clrbig#TK_CONST_PFLOAT~    {TK_PINUM}f|{float_num}f
    ^clrbig#TK_CONST_PDOUBLE~   {float_num}
    ^clrbig#TK_CONST_PLONGDBL~  {TK_PINUM}ld|{float_num}ld
    ^clrbig#TK_CONST_PDECIMAL~  {TK_PINUM}m|{float_num}m
\end{Verbatim}

\subsubsection{结构类型}

结构将一系列变量组织成一个单一的实体。每个变量称为结构的成员。
结构用~struct~来声明,如下所示:

\ttfamily
\begin{lstlisting}
    struct PhoneBook
    {
        int a;
        int b;
    }
    PhoneBook t;
\end{lstlisting}
\CJKfamily{song}

~t~就是一个结构类型的变量。结构的成员全部是~public~类型,即可以
通过变量名加上``~.~''号,再加上成员名称来访问:

\ttfamily
\begin{lstlisting}
    t.a = 2;
\end{lstlisting}
\CJKfamily{song}

结构的成员类型可以是前面的简单类型,也可以是结构体、联合体、枚举
或位段。结构体内部可以嵌套定义结构体、联合体、枚举和位段,其作用域
仅限于结构体内部。

正则表达式:

\begin{Verbatim}[frame=single, commandchars=^\#\~]
    ^clrsmall#letter~   [a-zA-Z_]
    ^clrbig#TK_KEY_STRUCT~  "struct"
    ^clrbig#TK_KEY_PRIVATE~ "private"
    ^clrbig#TK_KEY_PUBLIC~  "public"
    ^clrbig#TK_IDENTIFIER~  {letter}({letter}|{digit})*
    ^clrbig#TK_LCURLY~  "{"
    ^clrbig#TK_RCURLY~  "}"
    ^clrbig#TK_COMMA~   ","
    ^clrbig#TK_SEMI~    ";"
\end{Verbatim}

\subsubsection{枚举类型}

枚举~(enum)~为一组在逻辑上密不可分的整数值提供便于记忆的符号。可
如下声明:

\ttfamily
\begin{lstlisting}
    enum WeekDay
    {
        Sunday, Monday, Tuesday, Wednesday, Thursday,
        Friday, Saturday
    }
    WeekDay day;
    day = Tuesday;
\end{lstlisting}
\CJKfamily{song}

注意:结构是由不同类型的数据组成的一组新的数据类型,结构类型的变量
的值是由各个成员的值组合而成的。枚举类型的变量在某个时刻只能取枚举
中某一个元素的值,如上面的~day~,它的值要么是~Sunday~,要么是
~Monday~或其他的星期元素。但它在一个时刻只能代表具体的某一天,不能
既是星期二,又是星期三。

默认枚举中每个元素类型都是~int~型,且第一个元素的值为~0~,后面的
每一个连续的元素的值按加~1~递增。在枚举中,也可以给元素直接赋值,
如下代码把星期天的值设为~1~,其后元素的值分别为~2~,~3~$\ldots$。

\ttfamily
\begin{lstlisting}
    enum WeekDay
    {
        Sunday=1, Monday, Tuesday, Wednesday, Thursday,
        Friday, Saturday
    }
\end{lstlisting}
\CJKfamily{song}

为枚举类型的元素所赋的值的类型限于~long~、~int~、~short~和~byte~等
整数类型。

正则表达式:

\begin{Verbatim}[frame=single, commandchars=^\#\~]
    ^clrbig#TK_KEY_ENUM~    "enum"
    ^clrbig#TK_EQ~          "="
\end{Verbatim}

语法:

\begin{Verbatim}[frame=single, commandchars=^\#\~]
    ^clrsmall#enumerator_def~
     : TK_KEY_ENUM TK_IDENTIFIER
       TK_LCURLY ^clrsmall#enumerator_list~ TK_RCURLY
     ;
    ^clrsmall#enumerator_list~
     : ^clrsmall#enumerator~
     | ^clrsmall#enumerator_list~ TK_COMMA ^clrsmall#enumerator~
     ;
    ^clrsmall#enumerator~
     : TK_IDENTIFIER
     | TK_IDENTIFIER TK_EQ ^clrsmall#const_expression~
     ;
\end{Verbatim}

\subsubsection{联合体和位段类型}

联合体类似于

\subsubsection{小结}

前述结构体的对齐,
联合体
位段对齐。

前述的几种类型定义的前面还可以跟关键字~private~和~public~,表明
该类型可否被其他平级作用域所引用。

前述结构体、联合体、位段的语法:

\begin{Verbatim}[frame=single, commandchars=^\#\~]
    ^clrsmall#struct_def~ :
\end{Verbatim}

\subsection{引用类型}

引用类型的变量指该变量不直接存储所包含的值,而是指向它所要存储的值。
它存储的是实际数据的引用地址。引用类型本身不能再有引用类型,即引用的
引用是不允许的。

引用类型包括直接定义、委托和数组。

\subsubsection{直接定义}

任何数据类型包括派生类型都可以定义对应的引用变量。但是引用类型本身
不能定义对应的引用类型。如果有类似的编程需求,则可以将引用类型定义
在结构中,然后使用结构的引用变量。

\ttfamily
\begin{lstlisting}
    ref int a = null;
    int b;
    ref int c = ref(b);
\end{lstlisting}
\CJKfamily{song}

和~C++~不同,~W~语言的引用可以指向空对象。~ref~是一个取引用的操作符,
它可取得变量的引用地址并赋值给引用变量。

正则表达式:

\begin{Verbatim}[frame=single, commandchars=^\#\~]
    ^clrbig#TK_KEY_NULL~   "null"
    ^clrbig#TK_KEY_REF~    "ref"
\end{Verbatim}

\subsubsection{委托}

在~C~和~C++~语言中,指针是它们最强有力的工具之一,同时又给他们带来
很多苦恼之处。因为指针指向的数据类型可能并不相同,如你可能把~int~
类型的指针指向一个~float~类型的变量,而这时程序并不会出错。若你
删除了一个不应该被删除的指针,程序就有可能崩溃。可见,滥用指针给
程序的安全性带来隐患。

正因为如此,在~W~语言中取消了取消了指针这个概念。并且用引用来替代。
尽管如此,为了使~W~语言能编写最底层的系统程序,如驱动程序,~W~语言
仍然提供了语法,使得引用可以象指针一样操作,本身也可以参加运算和被
强制转换。引用变量后也可以跟数组运算符号~[]~,因此可以当作数组名
处理。这样的代码是不安全的,因此会在编译时作特别的提示。

委托~(delegate)~相当于~C~语言中的函数指针原型。委托是类型安全的。
委托的声明类似一般的函数声明,前面加~delegate~关键字,如:

\ttfamily
\begin{lstlisting}
    delegate int MyDelegate();
\end{lstlisting}
\CJKfamily{song}

以上声明了一个~MyDelegate~类型,在使用中需要利用这个类型来定义
变量,并把合适的函数地址赋给它。委托变量只能和~int~变量相互
强制转换。

\ttfamily
\begin{lstlisting}
    MyDelegate d;
    public int NormalMethod()
    {
        return 0;
    }
    d = NormalMethod;
    d();
\end{lstlisting}
\CJKfamily{song}

正则表达式:

\begin{Verbatim}[frame=single, commandchars=^\#\~]
    ^clrbig#TK_KEY_DELEGATE~ "delegate"
\end{Verbatim}

\subsubsection{数组}

数组是进行批量处理数据的。数组是一组类型相同的有序数据。数组
按数组名、数据元素的类型和维数来进行描述。

在~W~语言子集中,我们只实现固定长度的数组。固定长度的数组
定义格式:
\begin{Verbatim}[frame=single, commandchars=^\#\~]
    non-array-type[dim_separators] array-instance-name;
\end{Verbatim}

例如:
\ttfamily
\begin{lstlisting}
    int[5] a;
\end{lstlisting}
\CJKfamily{song}

实现时将在栈上分配相应的内存。如果要动态分配内存,则使用
~walloc~函数,并将返回值赋给一个引用变量。如下所示:
\ttfamily
\begin{lstlisting}
    ref int a = walloc(3*sizeof(int)) as ref int;
\end{lstlisting}
\CJKfamily{song}

使用数组时,可以在~``[]''~中加入下标来取得对应的数组元素。
数组元素的下标是从~0~开始的。第一个元素对应的下标是~0~,以后
逐个增加。

数组可以是多维的。而一维数组用的最普遍。下面是一个例子:
\ttfamily
\begin{lstlisting}
    int[5] arr;
    for(int i = 0; i < 5; i ++)
        arr[i] = i * i;
\end{lstlisting}
\CJKfamily{song}

这个程序创建了一个有~5~个~int~型元素组成的数组,并逐个赋值
初始化。下面是多维数组的声明和例子:
\ttfamily
\begin{lstlisting}
    int[2,3]   a;
    int[4,2,3] b;
\end{lstlisting}
\CJKfamily{song}

和~C~语言不同的是,各维数在一个~``[]''~内指明,用逗号隔开。使用
多维数组时也使用相同的格式。其中第一个维数变化慢,最后一个元素
变化快。对于二维数组,第一维就是行数,第二维就是列数。在内存中
的存储布局是按行存储的。大于三维时,最后二维看成面,前面的维数
指定了后面的面阵,并顺序存储。例如前例中的三维数组,在内存中的
排列次序为:
\ttfamily
\begin{lstlisting}
    b[0, 0, 0]  b[0, 0, 1]  b[0, 0, 2]
    b[0, 1, 0]  b[0, 1, 1]  b[0, 1, 2]
    b[1, 0, 0]  b[1, 0, 1]  b[1, 0, 2]
    b[1, 1, 0]  b[1, 1, 1]  b[1, 1, 2]
    b[2, 0, 0]  b[2, 0, 1]  b[2, 0, 2]
    b[2, 1, 0]  b[2, 1, 1]  b[2, 1, 2]
    b[3, 0, 0]  b[3, 0, 1]  b[3, 0, 2]
    b[3, 1, 0]  b[3, 1, 1]  b[3, 1, 2]
\end{lstlisting}
\CJKfamily{song}

多维数组前面部分维数可以看成一个引用变量。例如前面例子中的~a~是
~$2\times 3$~的数组,那么可以把~a[0]~和~a[1]~都看成引用变量,同时
也是一维数组名,数组长度都是~3~。上述数组名~a~或者~a[0]~可以被
强制转化为基本类型的引用变量,从而可以象指针那样高效率地访问内存
数据。

注意:数组定义不能用~typedef~来取得它的别名。
