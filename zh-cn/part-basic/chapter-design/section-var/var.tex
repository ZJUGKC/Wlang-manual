%
% Variables, XIN YUAN, 2020
%

每一门编程语言都有自己对变量和常量的命名和使用方式。本节阐述变量
和常量的用途、如何对变量和常量命名、如何定义和初始化变量和常量。

\subsection{变量}

程序要对数据进行读写运算等操作,当需要保存特定的值或计算结果时,
就需要用到变量。在变量中可以存储各种类型的信息,如姓名、车票价格
和文件长度等。

计算机中的变量代表存储地址,变量的类型决定了存储在变量中的数值的
类型。变量的值可通过赋值和“++”或“--”运算符的运算被改变。

变量使用的重要原则是:先定义后使用。

变量可以在定义时被赋值,也可以在定义时不被赋值。定义时被赋值的
变量具有初始值,而定义时不被赋值的变量没有初始值。可以在其后的
代码中为这样的变量赋值。
