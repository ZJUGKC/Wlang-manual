%
% Introduction, XIN YUAN, 2020
%

\chapter{W~语言设计介绍}

W~语言是作者提出的一个通用编程语言。它的设计动机是因为目前软件业
发展很快,涌现出了各种通用程序语言,如~C,C++,Java,VB,C\#~等等,
它们当初是根据软件业发展历史中各自不同的需求产生的,因此有它们各
自的特点,也有各自的局限性。随着开源软件的观念的深入,软件呈爆炸
性的增长,因此个性化编程的时代已经来临。程序员们不愿意被现有通用
语言的局限性所束缚,制作一个自己喜欢的,并且结合现有编程语言甚至
编程模型的各种优点以及降低它们的局限性的个性编程语言,是一种非常
有趣和有用的工作。本书就是基于这样的思想,设计了~W~语言,作为一个
个性化的编程平台的尝试。\index{个性化}

\section{当前通用语言特点}

目前的通用语言主要有面向过程和面向对象两大类。而当前面向对象的
语言成为主流。下面简单介绍各种语言的特点和局限性。
\begin{itemize}
\item{C}

C~语言是伴随着~Unix~出现的通用编程语言。它是面向过程的,自从出现
以后就受到各个领域的程序员的喜爱,从系统程序到应用程序,到处都可
以看到用~C~写的代码。~C~语言主要优点是语法简单,结构清晰,运算符
号丰富,适合编写系统级程序。几乎所有的运行在不同硬件平台上的操作
系统都提供~C~运行库。缺点是对于复杂的应用,使用~C~语言就非常麻烦,
出错的概率增大。

\item{C++}

C++~语言是从~C~语言发展而来,它增加了面向对象的特性,同时也兼容
面向过程的编程。因此使得~C++~语言不仅能编写系统级的软件,也可以
编写复杂的应用程序,比如数据库应用。缺点是语法过于复杂,语言过于
灵活,不仅难以掌握,而且写出的程序~bug~多,并且难调试。

\item{VB}

VB~是微软的快速开发工具,语言简单,容易掌握。并且有可视化的编程
工具,和组件化的开发方式,因此编写大型复杂的应用程序非常方便。缺
点是程序代码效率不高,不适合于实时性要求高的应用场合。和~VB~类似
的工具还有~Delphi~和~C++Builder~等。

\item{JAVA}

JAVA~是~C/C++~的发展,去掉了~C/C++~中的一些繁琐并且容易出错的
特性,并引入了现代的面向对象的特征,包括虚拟机和垃圾收集。~JAVA~
语言具有跨平台的特征,一次编写,到处运行。缺点是使用了虚拟机技术,
运行速度和效率较低。

\item{C\#}

C\#\cite{cs:2004}~是和~JAVA~竞争过程中发展出来的语言,直接建立在
~Windows~的~.Net~应用框架的基础上,同样引进了虚拟机和垃圾收集这些
现代编程语言特征。它非常适合网络应用的开发。和~JAVA~类似,同样有
运行速度和效率较低的缺点。

\end{itemize}

由此可见,目前的通用编程语言几乎都有一定的局限性,不能适应相当大
范围的应用开发的需求。

\section{语言设计构想}

本书要设计的语言应包含以下特征:

\begin{itemize}
\item{面向过程}

在某些应用中,只使用面向过程的设计方式已经足够。因此要设计的语言
应该包含这个特征。同时做一个改进,把处理同一类概念的过程集中放在
一起。\index{面向过程}

\item{面向对象}

面向对象是当前通用程序语言的主流特性,连~Linux~使用~C~语言编写
系统程序也使用了类似的面向对象的思想。面向对象主要应该包括封装
性、继承性、多态性。\index{面向对象}

\item{面向组件}

面向组件是另一种现代的模块化构造应用的方法。它采用接口的概念
作为模块的界面,接口提供一组函数,在面向对象中可以用一个纯虚
基类表示。组件接口的说明和实现是分离的。\index{面向组件}

\item{面向窗口界面}

窗口界面是现代操作系统的重要部分。因此要提供窗口界面的编程模型,
计划提供两层模型,一个是底层模型,基于窗口句柄和操作系统的接口,
另一个是容器-窗口组件模型,类似类似于微软的~ActiveX~和~Delphi~
的~VCL~组件方式。

\end{itemize}

\section{语言设计的基本布局}

根据以上的分析,作者提出了~W~语言,它以~C/C++~为蓝本,吸收了
其他各派语言之长,抛弃一些局限性,形成一种个性化的通用编程语言,
主要用于从系统软件到一般桌面应用程序,数据库应用,工业实时数据
采集和控制的应用等等这样一个较宽的范围。在这里,暂不考虑网络
应用和~Web~服务应用。其他要考虑的是跨平台的能力,包括硬件的不同
~(~大端或者小端数据格式,指令集的不同~)~,操作系统的不同~(~主要
是~Windows~和~Unix/Linux~系统,以下简称~X~系统~)~。

现在“设计模式”这个名词已经很普遍了,它是一种设计软件的方法。它的
提出者给出了~20~多种设计模式,其实常用的也就几种。当然,我们设计
的这个语言也要融合这些设计模式,并且成为语言特性的一部分,在我
看来是十分必要的。可以说,把设计模式结合到编程实践的最好方式就是
定制自己的语言。这样做更容易扩展自己的语言,同时也可以跟上现代
通用编程语言的演化步伐。

使用个性化通用编程语言写程序的一个潜在缺点是无法和其他写通用语言
的程序员交流。我的解决方法是提供一个双向转换语言的工具,即把~W~
语言转化为标准的~C++~语言,此外从现有的各类编程~API~(~如Windows的
函数库,~C/C++~运行库,~Windows~下的~COM~和~ActiveX~等等~)~转化
为~W~语言也是必须的。同时必须注意的是~W~语言必须提供一个可扩展的
通用类型系统,这样就可以和各种其他编程语言交流。

\chapter{开始编写程序}

编程风格:类的设计和定义全部放在~.h~中。
一个工程只有两个~.cpp~文件 尽量少用~.cpp~

字符串~\_T~的问题

由此带来的一些问题和解决方法
~static~变量的初始化。关键字
类的相互交叉引用 先定义 其后用~inline~

争议的技术:模板和多重继承
优缺点

应用程序:
模板作为派生类 作为基类,作为数据成员类型。
Mix-in

内联方法和内联的函数风格,自动决定内联,为什么这么做?
优缺点 强制非内联 noinline
\_\_forceinline

编辑器有展缩功能 因此都写一起没关系

.h写法参考ATL WTL 已有头文件写法,基本要素和例子举出。

如何自定义窗口消息?
