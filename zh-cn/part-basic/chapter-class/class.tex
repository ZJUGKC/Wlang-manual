%
% Class, XIN YUAN, 2020
%

\chapter{类和对象}

\begin{lstlisting}

# WLang语言

WLang语言是根据C++语言来设计的,吸收了C#、Java等语言的优点,专注于高层概念的构建,
适合多人协作开发,同时保持了C++高效执行的优点。

## 统一编程模型

计算机技术发展至今,已经深刻地改变了世界。计算机程序是计算机得以运转的基本要素。
计算机程序的运行本质上就是和计算机对话,向计算机要求计算资源,得到结果并呈现出来。
计算资源是一种服务,包括数据和算法,对应到计算机硬件的内存、硬盘 (它们存储数据)
以及CPU指令流 (它们执行算法)。数据和算法概念上是一体的,都是计算资源服务,
所以在软件设计上使用类来表示计算服务是合理的,因为类一般同时包含了数据和算法。
当然一个类也可以只包含数据,也可以只包含算法,并可以形成递归和嵌套。
WLang语言正是以类为核心来设计的。

由于计算资源服务统一被设计为类和类的实例对象,所以全局对象和孤立的函数失去了存在的必要,
可以用类的静态成员、静态方法来取代。这么做同时也是对全局对象和函数在概念上进行归类。

## 语法要素

### 注释

WLang的注释和C++相同,有/**/和//两种方式。

### 基本数据类型

WLang的基本数据类型如下表所示:

| 类型 | 说明 |
|-----|-----|
|char| 有符号字符类型,1个字节 |
|byte| 无符号字符类型,1个字节 |
|short| 有符号短整数,2个字节 |
|ushort| 无符号短整数,2个字节 |
|int| 有符号整数,4个字节 |
|uint| 无符号整数,4个字节 |
|int64| 有符号64位整数,8个字节 |
|uint64| 无符号64位整数,8个字节 |
|float| 单精度浮点数,4个字节 |
|double| 双精度浮点数,8个字节 |
|intptr| 指针类型的有符号整数,32位系统下是4字节,64位系统下是8字节 |
|uintptr| 指针类型的无符号整数,32位系统下是4字节,64位系统下是8字节 |
|bool| 布尔类型,占1个字节 |
|guid| 全局唯一标识符,占16个字节 |

其中,bool类型的取值是两个标准值:true和false,bool类型的变量本身就是逻辑表达式,
取值为false时为假,反之为真。

### 复合类型

由基本类型或复合类型组合的数据类型。WLang有两种复合类型:数组和类。数组本质上也是
一种“类”,不过是按重复模式构造出来的类型。类是一种用户可自由扩展定义的模式。基本类型
也可看成是“类”,因而类 (class) 是统一的类型。

由于需要对数组的元素进行各种操作,有可能对同一种元素要施行不同的操作,如对字符串类型
进行大小写敏感和不敏感的比较,因此将数组类型定义为特殊模板形式。

数组可如下定义:

```
//固定长度的数组
A[10] a;
A[10]<SpecialCompareTrait<A>> a;
//动态数组
share A[] a;
weak A[] a2;
```

枚举是一种特殊的类型,本身是整数类型,但只能取规定好的值的集合。枚举可如下定义:

```
enum StreamOpenFlags
{
	Read = 0x00000001U;
...
}
```

类和接口将在下面详述。

### 常量

每种类型都可能有常量,除了share类型和share数组。基本数据类型的常量如下表所示:

|类型|说明|
|---|---|
|char,short,int,int64,intptr|数字|
|byte,ushort,uint,uint64,uintptr|数字+后缀U|
|float|数字+小数点+数字+可选的指数部分+后缀f|
|double|无后缀浮点数表示|
|guid|{8位数字-4位数字-4位数字-4位数字-12位数字}|

常量用十六进制表示时,使用0x前缀。

常量可如下定义:

```
const int c_Num(5);
```

复合类型常量可以用花括号列表来定义,项之间可以用逗号隔开。花括号列表可以嵌套。

对于常量数组,其定义如下:

```
const int[] c_arr = { 0, 1, 2 };
```

常量可以定义在类中,类似static,也可以定义在方法体中。

字符串常量使用一对引号来定义。

转义字符:

|符号|描述|值|
|---|---|---|
|\a| 响铃(BEL) | 007 |
|\b| 退格(BS) | 008 |
|\f| 换页(FF) | 012 |
|\n| 换行(LF) | 010 |
|\r| 回车(CR) | 013 |
|\t| 水平制表(HT) | 009 |
|\v| 垂直制表(VT) | 011 |
|\\| 反斜杠 | 092 |
|\'| 单引号 | 039 |
|\"| 双引号 | 034 |
|\0| 空字符 | 000 |
|\ddd| 1至3位八进制数 | 3位八进制数 |
|\xhh| 1至2位十六进制数 | 2位十六进制数 |

### 名字空间

名字空间把类按概念组织成层次结构,自身也可以嵌套地定义。
名字空间可以看成是一种退化的类。
WLang语言规定,其类定义都要处于一个名字空间中。

名字空间如下定义:

```
namespace UI {
...
}
```

### 类和对象

类是提供服务的基本单位,是统一的编程模型。对象是类的实例。

类本身在名字空间中具有一个修饰说明,可以是private和public两种之一,分别代表
名字空间内可见以及名字空间外可见。

在类中,可以定义嵌套类,从而形成概念上的层次结构。

类由属性、方法、事件三种元素组成。属性和方法具有三种可见性修饰,private、protected
和public。事件由一个或多个接口组成,用event关键字说明。触发事件使用fire关键字。

成员对象和静态成员对象。

方法的声名末尾可以加上nothrow修饰,以声明它不会抛出异常。

方法可以带有static修饰符,以声明它是静态的,和类的实例无关。

方法的参数具有in, out, inout修饰,传递的是引用。返回值是对象的值传递。

方法内也可以嵌套定义类。

继承关系,允许多重继承,允许多个基类相同,此时一定要加上别名说明,调用方法使用别名
和::作为域限制符。覆盖实现基类的某个方法时,也可以加上域限制符,并使用override修饰符。

只有公有继承。

类和类的方法都可以设计成模板形式,模板参数可以是类型、整数常量和枚举常量。

类的对象除了直接定义外,还可以定义成引用ref T,共享对象share T以及弱对象weak T。
ref T的获取和传递要确保生命周期内的有效性。这三类引用对象都不能直接定义对象,只能通过
包装类的属性来间接访问,属性可以设计引用属性(ref,share,weak)和自身属性(包括数组类型)
两种。share T和weak T不能获得ref T对象。

对象的内存模型:
类一般通过对象实例来运行,除了静态成员方法。
类对象运行时将占据内存,其数据部分在使用中形成内存布局,除了静态成员对象。

|进程内存管理|描述|
|---|---|
|栈|方法内定义的变量,包括一般对象、引用对象、共享对象和弱对象|
|堆|共享对象、弱对象和部分引用对象管理的内存|
|全局数据段|静态对象或静态常对象、类的静态成员变量|
|全局未初始化数据段|方法内的静态对象|

对象的运行模型:
在程序运行时,对象所管理的数据可以不局限于内存,也可以是设备上的内存,如显卡显示内存等。
对象也可以管理磁盘文件等永久存储数据,或者虚拟概念上的存储数据。

|进程内|描述|
|---|---|
|内核空间|各种内核对象,如共享内存对象。暂时不支持驱动程序设计|
|装配件|对象占用所在进程的堆和栈|
|可执行装配件|对象占用所在进程的堆和栈|

|进程外|描述|
|---|---|
|远程装配件|代理对象占用所在进程的堆和栈,服务对象占用本地或远程计算机host进程的堆和栈|

对象的复制,等号或者拷贝构造、转移构造等全部缺省实现。引用对象缺省行为是被管理对象,
如果要对引用本身操作,使用ref()操作符来获得引用语义(浅拷贝)。如果要复制一个对象,使用clone
操作符(深拷贝)。判断是否空对象,使用is_null()操作符。

数据类型转换,as操作符。转换到其他基本类型;转换到基类;转化到派生类(通过模板参数);
ref,share,weak的直接转换。组件对象也可以有as,内部实现为query。

可以自定义构造方法。一般构造、拷贝构造、转移构造、赋值操作符、转移赋值操作符、析构均自动实现。

返回值可以是`weak ref`,用于封装第三方库的时候。

异常

### 接口

接口内包含属性和方法,无需进行可见性说明,它们均为公开的。接口由interface关键字来定义。

派生类可以以接口为基类,实现接口方法,包括属性的方法时,需要加上override修饰。

实现回调机制。

### 组件

进程内组件和进程外组件,局域网组件,网络组件。

支持as操作符,获得组件支持的其他接口。

实现多种事件集,连接点。观察者模式。

散集和列集。

实现插件系统。

## 基础类库


### 数据结构

System名字空间,允许intptr,uintptr和ref T的转化。使用MemoryPool基础类。本身符合
生命周期规定。

迭代器,数组和其他数据结构均可以有。使用auto_type关键字定义迭代器对象。

### 线程和同步

在应用层做。

线程间同步

几种线程模型

### 进程和同步

进程间同步

进程间通讯

## 工程类型

### 应用

### 类库

类库文件名带有版本号。

\end{lstlisting}
