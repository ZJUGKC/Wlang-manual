%
% Setup, XIN YUAN, 2020
%

\chapter{GKC安装和使用}

Wlang语言是GKC项目的一部分,所以需要先安装GKC。

\section{安装}

在 https://github.com/ZJUGKC/GKC 上下载源代码,编译生成Windows和Linux下的Release安装包。
将安装包和setup/uninstall脚本 (在windows下是setup.vbs和uninstall.vbs,
在Linux下是setup.sh和uninstall.sh) 打包成一个发行包文件。

将发行包解压缩,使用适当的参数运行setup脚本。这时系统中增加了一些环境变量,
其中三个环境变量指向自动创建的或者是参数指定的三个目录,构成了一个虚拟文件系统。
这三个目录如表 \ref{tab:setup:directories} 所示:

\begin{table}[h]
  \centering
  \caption{目录列表}\label{tab:setup:directories}
\begin{tabular}{|l|l|}
  \hline
  % after \\: \hline or \cline{col1-col2} \cline{col3-col4} ...
  \textbf{目录名} &  \textbf{描述} \\
  \hline\hline
  /system &  系统文件 \\
  \hline
  /lws &  本地工作空间,存放安装在本机的软件包 \\
  \hline
  /uws &  通用工作空间,存放用户工作数据,可以是共享文件夹 \\
  \hline
\end{tabular}

\end{table}

\section{使用}

在系统的命令行界面中使用Wlang的编译命令工具。
