%
% Setup, XIN YUAN, 2020
%

\chapter{WTL~的安装和使用}

WTL~\index{WTL}是微软的~ATL~\index{ATL}库的一个扩展,
并未得到微软官方的支持。但是在微软网站上,该库仍然在
不断地更新。作为一种方便地构造桌面程序的框架库,它
仍然得到关注。

\section{安装}

先到微软网站上免费下载~WTL~。本书使用~WTL8.0~,在
~Visual Studio 2005~\index{Visual Studio}
下使用。


将压缩包解压缩,假设解压目录为~WTL80~,将此目录移至
~Visual Studio 2005~的安装目录下的~VC~目录下,和
~ATLMFC~处于同一个层次的目录下。然后打开~Visual Studio
2005~,在~Tools-$>$Options~菜单下找到~Projects and
solutions~页,其中的~VC++ directories~下的~Include
files~增加~WTL80/include~所在的路径。


下一步是安装~Wizard~。找到~WTL80/AppWiz~,双击
~setup80.js~文件,此时再打开~Visual Studio 2005~,
新建一个工程,就可以发现多了一个~WTL~目录夹,其中有
~ATL/WTL Application Wizard~图标。利用此模板可创建
~WTL~应用程序。


\section{使用}

新创建一个解决方案\index{解决方案},利用~ATL/WTL~的
~wizard~\index{wizard},应用类型选择~Dialog Based~和
~Modal Dialog~,含义是基于对话框的应用程序,对话框是
模式对话框。

应用程序的 wizard 为我们产生了一系列的文件,
重要的文件如表 \ref{tab:setup:files} 所示。

\begin{table}[h]
  \centering
  \caption{文件列表}\label{tab:setup:files}
\begin{tabular}{|l|l|}
  \hline
  % after \\: \hline or \cline{col1-col2} \cline{col3-col4} ...
  \textbf{文件名} &  \textbf{功能} \\
  \hline\hline
  stdafx.h &  预编译头文件 \\
  \hline
  stdafx.cpp &  预编译实现文件 \\
  \hline
  resource.h &  资源的定义文件 \\
  \hline
  [~工程名~].rc & 资源文件 \\
  \hline
  [~工程名~].vcproj & 工程文件 \\
  \hline
  [~工程名~].h & 空文件 \\
  \hline
  [~工程名~].cpp & 主文件 \\
  \hline
  MainDlg.h & 对话框头文件 \\
  \hline
  res/[~工程名~].ico & 像标文件 \\
  \hline
\end{tabular}

\end{table}

查看这些源文件的代码,会发现它们具有简单的结构,
优雅的语法。几乎都使用头文件来实现重要的功能,
.cpp~文件只有两个,一个是用于生成预编译头,一个
则是含入口主函数的文件。头文件全部包含在这两个
~.cpp~文件中。这样的风格类似于~Java~程序,即把
类的说明和实现放在了一起。因为~Visual Studio 2005~
的编辑器具有收放函数体的功能,因此编程时不会感到
文件过于庞大,还是可以接受的。

编译并运行这个工程。可以看到一个对话框的界面,
类似于使用~MFC~编写的程序。

\begin{description}
\CJKfamily{zhhei}
    \item [注意]
\CJKfamily{zhkai}
Release~版本链接时可能出现~CRT~库的重复定义错误,这需要在
工程的属性配置中去掉预定义宏~\_ATL\_MIN\_CRT~,在~Visual
Studio 2005~下将预定义宏的继承选择框去掉即可。重新编译就
可以通过。

\CJKfamily{zhhei}
    \item [注意]
\CJKfamily{zhkai}
在编译大型工程时,由于需要实例化的类太多,可能会出现
编译器的堆不足的错误。此时可根据错误提示,在编译器的
命令行中增加指令,提高编译器堆的大小的数值,就可通过
编译。

\CJKfamily{zhhei}
    \item [注意]
\CJKfamily{zhkai}
一般编译出来的程序依赖于~ATL80.DLL~和~MS~的~CRT~库。
要去除依赖性,打开~Project Properties~对话框,在左边的
树图中打开~Configuration Properties~分支,点击~General,
在右边的列表中将~Use of ATL~改成~Static Link to ATL,
将~Minimize CRT use in ATL~先改成~Yes,按确定按钮,
将工程编译一遍。然后再打开~Project Properties~对话框,
将~Minimize CRT use in ATL~再改成~No,按确定按钮,
再将工程编译一遍。此时再打开~depend~程序查看生成的
目标程序,就可以发现~ATL80.DLL~及~MS~的~CRT~库的依赖性
已去除。这种方法下无需再将预定义宏的继承选择框去掉。

\CJKfamily{zhsong}
\end{description}

ATL/WTL~的~wizard~还有其他一些工程类型,表
~\ref{tab:setup:project}~列出了它们的解释。
\begin{table}[h]
  \centering
  \caption{工程类型}\label{tab:setup:project}
\begin{tabular}{|l|l|}
  \hline
  % after \\: \hline or \cline{col1-col2} \cline{col3-col4} ...
  \textbf{类型} & \textbf{说明} \\
  \hline\hline
  SDI Application & 单文档应用 \\
  \hline
  Multiple Threads SDI
  & 多线程单文档应用~(~如~IE~浏览器~) \\
  \hline
  MDI Application & 多文档应用 \\
  \hline
  Tab View Application
  & Tab~视应用~(~类似~Visual Studio~编辑器~) \\
  \hline
  Explorer Application & 资源管理器应用 \\
  \hline
  Dialog Based & 对话框应用 \\
  \hline
\end{tabular}

\end{table}

工程的可选项如表~\ref{tab:setup:option}~所示。
\begin{table}[h]
  \centering
  \caption{工程可选项}\label{tab:setup:option}
\begin{tabular}{|l|l|}
  \hline
  % after \\: \hline or \cline{col1-col2} \cline{col3-col4} ...
  \textbf{选项} & \textbf{说明} \\
  \hline\hline
  Enable ActiveX Control Hosting
  & 允许嵌入~ActiveX~控制 \\
  \hline
  Create as a COM Server & 创建~COM Server \\
  \hline
  Generate .CPP Files & 生成~.CPP~文件 \\
  \hline
  Add Common Controls Manifest
  & 加入通用控制的构造文件 \\
  \hline
  Unicode Character Set & 使用~Unicode~字符集 \\
  \hline
\end{tabular}

\end{table}
